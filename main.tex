\documentclass{article}
\usepackage[utf8]{inputenc}

\title{The Impact of Dijkstra's Go To Statement Considered Harmful  }
\author{1902604 Alper Mimoglu}

\usepackage{natbib}
\usepackage{graphicx}

\begin{document}

\maketitle

\section{Introduction}

This paper is about my understanding of Dijkstra's Go To Statement Considered Harmful and what impact it had on the field of computer science and the quality of code with goto statement compared to without the goto statement. Along side my understanding and my point of view.
\section{What is the field of computer science?}

The field of computer science is the study of computers and computational systems. Unlike electrical and computer engineers, computer scientists deal mostly with software and software systems; this includes their theory, design, development, and application.\cite{dept_of_CS}

\section{Summary of Dijkstra's Go To Statement Considered Harmful}

This paper outlines why the goto statement should be taken out of modern day programming languages  because it is too “primitive” and it would make the code inefficient and it ruins programmers code \cite{dijkstra1968_goto} because the goto causes more problems then it solves, then he goes over that the  programmer's activity ends when he has constructed a correct program \cite{dijkstra1968_goto}
but for a programmer to create correct code they need to make it so it is sustainable to maintainable to avoid spaghetti code which can do the task of the program but it would make the program harder to maintain.


\section{The Impact of Dijkstra's Go To Statement Considered Harmful}
The impact that Dijkstra had on the field of computer was spreading his view about the goto statement and everybody in the field of computer science started not using the goto statement but found better alternatives and they improved coding by using while loops and if statements this made coding more complex but more cleaner and more functional in a sense that there is less spaghetti code because the goto statement is looked down upon by educators when they are teaching code because there is far more better and easier ways of doing the same function with a more stable end result and it is more maintainable because it is more easier to debug.

\section{Conclusion/My opinion}
I agree with Dijkstra that the goto statement should be abolished because the goto statement is not harmful how ever the "good" uses for the goto fall into a specific use cases\cite{Wulf:1972:CAG:800194.805861}, but majority of code that uses goto just turns into spaghetti code which is not great code and for programmers to get better at coding the goto statement needs to go too.



\bibliographystyle{plain}
\bibliography{references}
\end{document}
